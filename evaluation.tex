%!TEX root = ./report.tex
\section{Evaluation}
In order to test and evaluate the different parts of the project, we have created tests that check (1) the results of the merged algorithm and (2) our own isolated implementation of the trapezoidal map algorithm. We outline our test in this way to provide some confidence in the functionality of the overall solution and to identify the parts of it that are most likely to contain errors. 

Generally, we divided the tests into two categories and created an automated test case and a couple of static ones. Common to all test cases is that they feed a number of sites to the algorithm (or a number of line segments depending on the algorithm to be tested), performs a number of point location queries on the resulting search structure and compares the result to the expected output.

\subsection{Fortune’s algorithm}
We did not spend a lot of time testing the implementation of Fortune’s algorithm since we know its output is only valid to a certain degree. Using the GUI tool that came with the algorithm, we can see that line segments close the boundaries of the diagram are not included in the output. This means that the Voronoi diagram is corrupted and that a cell neighbour to the boundaries of the diagram can contain more than one site. This is a serious error that obviously causes a certain number of the failures in our tests.

\subsection{Automated test}
We created an automated test case that generates 1000 sites with random coordinates lying in the interval of [1-10.000] on both axis and feeds it to the merged algorithm. Since any site in a Voronoi diagram should obviously return its own Voronoi cell when used as query point in the search tree, this is exactly how we perform our test.

Due to fact that the merged algorithm check each line segment before it adds it to trapezoidal map, we know that the input set is in general position. On the other hand, if line segments that are not in general position occur, these will be removed from the Voronoi diagram and corrupt it even more. Still we can run our automated tests on any number of sites and get a success rate of approximately 20\% when perform the point location queries using the generated sites. This result is obviously not good but it is also not insignificant considering that the Voronoi diagram can potentially be seriously corrupted. 

Knowing that the implementation has trouble with generation of edges along the boundaries of the diagram it is also an interesting observation that the success rate of the automated tests tend to increase when we sort out query points from the problem areas.\paragraph{}

To convince ourselves that a substantial amount of the errors arise inside the implementation of Fortune’s algorithm, and in the filtering of the segments, we have tested our own implementation in an isolated environment 

\subsection{Static test}
While an automated test can generate a great amount of random test cases, we can only perform point queries with the sites from the input as these are the only points we can be sure to the know the Voronoi cells to. In the context of a trapezoidal map, we can generate random line segments and enforce their validity as input but we can’t write a procedure to check the output.

In order to evaluate the quality of our implementation of the trapezoidal algorithm, we have constructed a couple of limited and short test cases whose output have been manageable to check manually. Each test case contains a set of line segments forming a complete and valid Voronoi diagram and a number of query points who returned a correct result in each case. Besides these very small test cases, we invested a great deal of time in making sure that we handled the representation of the data structures, the representation of trapezoids (with pointers to neighbours etc.) and the search tree correctly.
Furthermore, we have run our implementation of the trapezoidal map algorithm many times during the last phases of development with successful results, so we are fairly confident that it works as intended.

The original thought was to run a final test experiment on a real world data set of fast food restaurants in the US. This proved not to be possible as the implementation of Fortune’s algorithm overflowed the stack after approximately 10.000 records of input. While it would be valuable to assert the applicability of the algorithm on a real world data set, we believe that the results from the remaining tests are good enough on their own. We find that the trapezoidal map algorithm seems promising enough combined with a better implementation of Fortune’s algorithm. 


