%!TEX root = ./report.tex

\section{Threats to Validity}

While we believe that our evaluation ultimately makes a valuable and sufficient test of the solution for the scope of this project, there are a couple of possible threats to validity.

\subsection{Internal Threats}

\begin{itemize}
	\item In the current set up we apply our solution to 5 data sets of limited size whose results are manageable to deal with outside the algorithm. They all produced a correct result which provides a great deal of confidence in the correctness of the implementation. A better test would, however, involve more test cases…

	\item Our solution is constructed from two separate and real-world algorithms for point location and Voronoi diagram generation. We implemented point location in C\# and found an implementation of Fortunes Algorithm online written in .NET 1.0. During the project we discovered that the latter implementation, in its original form, contained small errors and that some Voronoi cells of the output contained two or three sites instead of just one.

\paragraph{}
These erroneous cells are very small in number and always occur in positions adjacent to the boundaries of the diagram. The consequence of the error to our final product is that a lookup in the search tree with a query point contained in an erroneous cell will return a random site. In a real-world scenario the impact from this threat would be major but for the scope of this project we believe it is minor as we have identified the error and take it into account when designing our test cases.
\end{itemize}

\subsection{External Threats}

\begin{itemize}
	\item An ambiguity appears when performing point locations with query points that lie exactly on a line segment or a vertex. Reading through the literature listed in the references to this report, certain solutions are mentioned. Still we choose not to take this case into account since the assumption that line segments are in general position seems reasonable for the scope of this project and we still handle a pair of other similar ambiguities.

\paragraph{}
The general position assumption is however much less reasonable in the context of fast food restaurants as data from the real world is non-controllable. Therefore, not handling this case infers some restrictions on the generalizability of our results in a real world context which is a threat to validity. 

\end{itemize}
        
           