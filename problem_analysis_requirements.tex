%!TEX root = ./report.tex
\section{Problem Statement}
As mentioned in section, there are multiple useful applications of point location in voronoi diagrams. In the remainder of this report we describe a solution that facilitates this kind of point location using techniques from two algorithms that are known in advance – one that generates voronoi diagrams and one that performs point location in any planar subdivision. 

\subsection{Desirable Features}
\label{problem_analysis}

\begin{itemize}
  \item The complexity should be unchanged, i.e. not bigger than any of the existing algorithms
  \item The solution must be generic, i.e. not targeted a specific business area or domain
\end{itemize}

\subsection{Assumptions}
\label{assumptions}

\begin{itemize}
  \item We require that no two sites of the input share the same x-coordinate as this is a simplifying assumption made by the original point location algorithm. This assumption means that every trapezoid can be uniquely represented in terms of their left point, right point, bottom segment and top segment. Furthermore, every trapezoid can have at most four neighbours. It is clear that this assumption is unreasonable when working with real-world data. How to deal with input where such “degenerate” cases occur is described in \cite{computational_geometry}
\end{itemize}





